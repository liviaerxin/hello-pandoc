% from https://github.com/HaoZeke/ui_sens_phd_plan

\documentclass[a4paper,english]{article}

\usepackage{graphicx}

\usepackage{babel,csquotes}
\usepackage{fancyhdr}
\pagestyle{fancy}

\usepackage{xcolor}
\usepackage{enumitem}

%%%%%%%%%%%%%%%%%
% Bibiliography %
%%%%%%%%%%%%%%%%%

\usepackage{biblatex}

% get rid of the number-labels ([1], [2], etc.) in front of publications
\defbibenvironment{midbib}
{\list
	{}
	{
		\setlength{\leftmargin}{0mm}
		\setlength{\itemindent}{-\leftmargin}
		\setlength{\itemsep}{\bibitemsep}
		\setlength{\parsep}{\bibparsep}}
}
{\endlist}
{\item}

%%%%%%%%
% Temp %
%%%%%%%%
% Delete these in production
\usepackage[pangram]{blindtext}
\usepackage{lipsum}

%%%%%%%%%%%%%%
% Mimic Word %
%%%%%%%%%%%%%%

% Set header rules
\renewcommand{\headrulewidth}{0pt}
\usepackage[headheight=75pt]{geometry}
\setlength\headsep{20pt}

% Use the header
\chead{\includegraphics[width=\textwidth]{logo/sensCutout.png}} % Use image
\lhead{} % Clear left
\rhead{} % Clear right
% Manipulate fonts
\usepackage{fontspec}

\setmainfont[
	Path = Fonts/Calibri/,
	Numbers=OldStyle,
	Extension = .ttf,
	UprightFont = *_Regular,
	ItalicFont = *_It,
	BoldFont = *_Bold,
]{Calibri}

% Allow centering of sections
\def\centersec#1{\centering#1} % kanged from https://tex.stackexchange.com/questions/193629/how-to-center-an-unnumbered-section-name

\title{UI SENS PhD Study Plan}

% Change these
\author{Jane Doe}
\addbibresource{smplBibtex.bib}

\begin{document}

\section*{\centersec{PhD Study Plan}}

\begin{description}[itemsep=1pt]
  \item[Name:] Jane Doe
  \item[Kennitala:] 020593--5291
  \item[Degree Program:] Chemistry
  \item[PhD studies start date:] 9\textsuperscript{th} of January 2020
\item[Doctoral Committee:] \quad (Not) Appointed by the Faculty Council \\
  \hspace*{75pt} Prof.\ Person1 Lastname1, Designation, University of Iceland (supervisor) \\
  \hspace*{75pt} Prof.\ Person2 Lastname2, Designation, University of Iceland \\
  \hspace*{75pt} Dr.\ Person3 Lastname3, Designation, Affiliation \\
  \item[Funding status:] Lots of money from blah
  \item[Midway evaluation:] Not completed, tentatively in the Spring of 2021
  \item[Midway presentation:] Not completed, tentatively in the Spring of 2021
  \item[Student backgroud:] \hfill \\ \Blindtext[2][3]
\end{description}

\subsection*{Prior publications}
\nocite{*}
\textbf{Articles}
\printbibliography[type=article, env=midbib, heading=none]
\noindent\textbf{Books}
\printbibliography[type=book, env=midbib, heading=none]

\subsection*{Research Project(s)}
% Repeat if necessary
\subsubsection*{\centersec{Awesome Project I}}
\begin{refsection} % New section for references
\lipsum[1-5]{}
\nocite{*}
\printbibliography[heading=subbibliography] % print section bibliography
\end{refsection}

\subsection*{Time Schedule}
\subsubsection*{First Year}
\lipsum[28]{}
\subsubsection*{Second Year}
\lipsum[77]{}
\subsubsection*{Third Year}
\lipsum[63]{}

\subsubsection*{\centersec{Estimated Defense Date: August 2023}}

\subsection*{Doctoral Dissertation}
The thesis will consist of a series of peer reviewed papers published in leading journals within the interdisciplinary fields relevant to the projects described. The thesis book shall also include a self-contained introduction and coherent rationale for the project and state of the art techniques to put the work in context. A tentative table-of-contents is as follows:

\begin{itemize}[noitemsep]
  \item Motivation
  \item Introduction
  \item Methods
  \item Results
  \item Conclusions and Future Directions
  \item Summary of Papers
  \item Full-text of papers
  \item References
\end{itemize}

\begin{description}[itemsep=1pt]
  \item[Prerequisite courses:] Total Credits: 0
\item[Course Plan:] \quad Total Credits: 32.5 \\
  \hspace*{75pt} \textit{Autumn 2018} \\
  \hspace*{75pt} EFN115F Computational Chemistry \\
  \hspace*{75pt} \textit{Summer 2020} \\
  \hspace*{75pt} 1023 CAMD Summer School \\
\end{description}

\noindent All course descriptions are presented in an appendix.
\newpage
\noindent In Reykjavik, \today\\[3cm]
\noindent Student \hspace*{\fill} Advisor \hspace*{\fill} Committee members\\
\noindent Jane Doe \hspace*{\fill} Person1 Lastname1 \hspace*{\fill} Person2 Lastname2\\
\hspace*{\fill} Person3 Lastname3\\[1cm]
\noindent \textbf{Confirmed by:}\\[3cm]
Faculty Graduate Studies Committee Representative

\newpage
\section*{Appendix - Course Descriptions}
\subsection*{Courses taken at the University of Iceland}
\subsubsection*{EFN115F - Computational Chemistry (10 ECTS)}
Introduction to methodology and tools for studying the structure of molecules, chemical bonding and chemical reactions. A survey of computational approaches for calculating electron distribution such as ab initio methods (Hartree-Fock, configuration interaction, perturbation theory), density functional theory (various functional approximations) and semi-empirical methods will be given. Methods for calculating dynamics of atoms in molecular vibration and chemical reactions. The goal is to make students capable of using research level tools and carry out simple calculations related to their research interests.
\begin{itemize}
        \item Basic concepts of quantum mechanics, variational calculations, Hartree- Fock approximation for electronic systems and basis sets
        \item Post Hartree-Fock methods (Moller-Plesset perturbation theory and configuration interaction) 
        \item Density functional theory (local density approximation, gradient dependent functionals, self-interaction correction)
        \item Semi-empirical methods
        \item Normal modes of vibration, transition state theory and harmonic transition state theory, tunneling.
\end{itemize}

\subsection*{External Courses}
\subsubsection*{1023 CAMD Summer School (5 ECTS)}
\lipsum[77]{}
\lipsum[28]{}
\end{document}